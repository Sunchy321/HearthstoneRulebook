\chapter{顺序和条件}

\section{效果顺序}

效果顺序一般与战吼/法术描述文字中所写出的顺序相同,但也存在不同的情况。

\subsection{双方效果}

描述为「双方各」的效果,一般是效果发动者优先执行。
\example \card{寒光智者}首先为其操控者抽两张牌。
\example \card{决斗}和\card{先祖召唤}首先为法术施放者召唤一个随从。

描述为「双方英雄各」的效果,一般对先入场的英雄首先生效。
\notice 如果双方均未替换过英雄,主玩家的英雄先入场。
\example 双方均未替换过英雄,\card{翡翠掠夺者}和\card{暗影子嗣}首先对主玩家造成伤害。
\example 双方均未替换过英雄,\card{亡灵药剂师}和\card{生命之树}首先为主玩家恢复生命。

\subsection{召唤随从并附加状态的效果}

一些效果在召唤随从的同时为其附加某个状态。事实上这类效果的结算方式可能是「先召唤随从并结算其召唤事件,再附加状态」或「先召唤随从并附加状态,再结算其召唤事件」,究竟采用哪种方式目前暂无明确规律。这导致了以下互动:

如果一个效果「先召唤随从并附加状态,再结算其召唤事件」,那么这个状态可以被\card{卡德加}复制。反之则不行。
\example 卡德加可以复制\card{小鬼妈妈}召唤的恶魔所具有的嘲讽,但不能复制\card{萌芽分裂}召唤的随从所具有的的嘲讽。

如果一个效果「先召唤随从并附加状态,再结算其召唤事件」且召唤的随从具有休眠,则该状态在苏醒时保留。反之则不行。
\example \card{巴内斯}召唤的休眠随从苏醒时仍为1/1,但\card{黑暗预兆}召唤的休眠随从苏醒时不能获得 +3 生命值。

但无论如何,光环更新总是最先进行的。
\example 你具有\card{水晶核心}的效果并使用巴内斯召唤了一个随从。水晶核心立刻为这个随从附加一个名为「\card{晶化}」的状态使其变为4/4,然后因巴内斯的效果变为1/1。
\example 你操控\card{暴风城勇士}并使用巴内斯召唤了一个随从。暴风城勇士立刻为这个随从附加一个 +1/+1的状态,然后因巴内斯为其附加一个变为1/1的状态。但因为暴风城勇士的状态的低优先级,这个随从首先因巴内斯的状态变为1/1,再因暴风城勇士的状态变为2/2。

\notice \card{救赎}等效果并不是添加状态,而是通过修改伤害值\texttt{.DAMAGE}来将当前生命值置为1。这个效果的时间点早于随从享受到场上的光环。
\example 你操控\card{暴风城勇士}并通过救赎复活了你的\card{冰风雪人}。雪人首先因救赎的效果,当前生命变为1,然后受暴风城勇士光环影响变为5/2。
\notice 如果你在暴风城勇士的光环下复制一个场上随从并将其当前生命值置为1(如通过\card{鲜活梦魇}),则复制不会变成2血。

\subsection{反常的结算顺序}

有的效果的描述与实际的流程是相反的。这可能是一个 bug。
\example \card{暗影烈焰}描述为「消灭一个友方随从,对所有敌方随从造成等同于其攻击力的伤害。」,但实际上却刚好相反。这就意味着,如果对方操控\card{飞刀杂耍者}和\card{小鬼首领},你对你的\card{小精灵}使用\card{暗影烈焰}:如果按照描述执行,小精灵会首先被标记为待摧毁,因此飞刀不会射到小精灵。但实际上飞刀可能射到小精灵。
\example \card{神圣愤怒}描述为「抽一张牌,并造成等同于其法力值消耗的伤害。」,但实际上却刚好相反。这保证了神圣愤怒按卡牌在牌库中的费用造成伤害,避免了其他效果的影响。
\example \card{命令怒吼}描述为「在本回合中,你的随从的生命值无法被降到1点以下。抽一张牌。」,但实际上却刚好相反。这导致了如果你刚好抽到了\card{破海者},你场上1血的随从无法因为命令怒吼的效果而存活下来。
\example \card{末日仪式}描述为「消灭一个友方随从。如果你拥有5个或更多随从,召唤一个5/5的恶魔。」,但实际上却刚好相反。这导致了如果你只拥有5个随从,也可以触发末日仪式的效果。以及消灭随从之前没有5个或更多随从,但是消灭随从后拥有了5个或更多随从也不会召唤5/5的恶魔。

卡扎库斯药水的不同组合有不同且确定的结算顺序。详见\card{卡扎库斯}。

\subsection{AoE的结算顺序}

有的AoE按照一个特定的顺序造成伤害,而不是一个群体伤害:
\begin{itemize}
    \item \card{横扫}首先对目标造成伤害,再对其他敌方角色造成群体伤害。
    \item \card{冰锥术}的伤害顺序是「左边,右边,中间」。
    \item 此外所有描述为「对目标及其两侧的随从造成伤害」的效果造成伤害的顺序都是「中间,左边,右边」。包括\card{爆炸射击}、 \card{强风射击}等。
\end{itemize}

一些 AoE 的实际结算顺序如下:
\begin{itemize}
    \item \card{暴风雪}首先对所有敌方随从造成2点伤害,然后冻结所有敌方随从。也就是说,因群体伤害而产生的随从也会被冻结。
    \item 与之相对的是,\card{冰锥术}首先对那几个随从造成伤害,再冻结同样的随从。它不会考虑是否产生了新的随从或是随从之间的位置关系发生了变化。
\end{itemize}

\section{扳机顺序}

前面提到,响应同一个事件的绝大多数扳机按入场顺序触发。实际上,这句话只描述了扳机都在场上的情况。如果扳机位于各个不同区域,情况会有所不同。此外,部分扳机还具有特殊的优先级。

\subsection{不同区域扳机的结算顺序}
\label{trigger-order-in-different-zone}

不同区域的实体具有的扳机可以划分为场上扳机、手牌扳机和牌库扳机三种。
\begin{itemize}
    \item 奥秘、武器、法术和场上的随从具有的扳机都是场上扳机。
    \item 状态具有的扳机都是场上扳机,即使这个状态是结附在手牌或牌库的牌上。
    \item 手牌中的随从具有的扳机是手牌扳机,例如\card{通道爬行者}。
    \item 牌库中的随从具有的扳机是牌库扳机,例如\card{海盗帕奇斯}。
\end{itemize}

当一个事件发生时,所有可用的扳机按照场上、手牌、牌库的顺序触发。
\example 你使用\card{南海船工}。首先场上的\card{冒进的艇长}触发,然后手牌中的\card{空降歹徒}触发,最后牌库中的\card{海盗帕奇斯}触发,不管它们以什么顺序生成,也不管它们在什么时候进入战场/手牌/牌库。

当我们称一个扳机\term{在场}时,实际上是指它位于可触发的区域。如\card{伯瓦尔·弗塔根}在场指在手牌中。

随从自带的扳机(如\card{飞刀杂耍者})的操控者与随从操控者相同,所处区域与随从所处区域相同,入场时间点与随从入场时间点相同。但由状态添加的扳机(如\card{力量的代价})的操控者为效果的使用者,所处区域始终视为场上,入场时间点为状态添加的时间点。
\notice \card{百变泽鲁斯}在未变形时是一个手牌扳机,而变形后是一个被附加了「\card[OG-123e]{变形} 」状态的随从,因而此效果对应的扳机属于场上扳机,入场时间点为上次变形时。其他类似的卡牌也有着相同的机制,包括\card[ICC-827t]{暗影映像} 。
\example 如果你的手中有未变形的\card{百变泽鲁斯}并使用\card{报警机器人},下个回合开始时报警机器人从手牌中召唤出泽鲁斯,它不能变形。而如果你的手中有已变形的泽鲁斯并使用报警机器人,下个回合开始时首先泽鲁斯继续变形,然后报警机器人从手牌中召唤出变形后的随从。
\example 你持有\card{软泥教授弗洛普}并使用\card{灵魂歌者安布拉}和\card{伊克斯里德,真菌之王}。软泥教授在真菌之王的使用后步骤中变形,因此其入场时间点晚于真菌之王(注意已变形的软泥教授是一个场上扳机,因此讨论入场时间点是有必要的)。你接着使用\card{破棺者}。在召唤后步骤中软泥教授尚未变形,因此安布拉不能将其召唤出来。在使用后步骤中首先真菌之王复制一个破棺者并由安布拉触发亡语,而此时软泥教授仍尚未变形,安布拉依然不能将它召唤出来。最后软泥教授变形为破棺者,序列结束。

\subsection{特殊优先级}
\label{special-priority}

部分扳机有特殊优先级,使它们与其他扳机之间不按照入场顺序和区域顺序结算。

\card{救赎}有低优先级,它总是晚于那些没有特殊优先级的场上死亡扳机。
\example 你操控救赎和\card{自爆绵羊}且持有\card{伯瓦尔·弗塔根}.对手杀死你的自爆绵羊。无论使用顺序如何,自爆绵羊的亡语首先触发,然后救赎召唤一个自爆绵羊,最后你的伯瓦尔获得 +1攻击力。

回响牌和\card{不稳定的异变}的消失有高优先级,它总是早于那些没有特殊优先级的回合结束扳机。
\example 你操控\card{基维斯}且持有回响状态下的不稳定的异变。回合结束阶段,首先不稳定的异变消失,然后基维斯为你抽满3张牌。

如果你一回合使用多张秘密通道,回合结束时所有秘密通道会按倒序结算,且以你使用第一张秘密通道之时为时机。
\example 你依次使用\card{唤醒}和秘密通道。回合结束时首先结算唤醒,但此时唤醒牌并不在你手牌中因此不会弃掉。然后秘密通道将那些牌返回你手牌。
\example 你依次使用秘密通道、唤醒和秘密通道。回合结束时首先结算两张秘密通道,将唤醒牌返回你的手牌。然后结算唤醒将那些牌弃掉。
\notice 上面的结算表明,只有你在使用唤醒之前没有使用过秘密通道的情况下,回合结束时唤醒牌才能保留。

\nameref{reborn}有最低优先级。它总是晚于一切其他死亡扳机。
\example 你持有\card{伯瓦尔·弗塔根}并杀死一个友方\card{鱼人木乃伊}。伯瓦尔首先增加1点攻击力,然后鱼人木乃伊才复生。

\nameref{lifesteal}和\nameref{overkill}有低优先级。它们总是晚于一切其他伤害扳机。
\example 你令你的\card{竞技场奴隶主}攻击敌方英雄并被\card{误导}到友方\card{龙蛋}。无论使用顺序如何,龙蛋首先触发召唤一条\card{黑色雏龙},然后竞技场奴隶主触发召唤另外一个竞技场奴隶主。
\notice 如果你使一个超杀法术获得吸血(通过\card{欧米茄灵能者}),超杀先于吸血生效;如果你使一个超杀随从获得吸血,则吸血先于超杀生效。

预伤害扳机之间有固定的触发顺序。详见\nameref{predamage-trigger}。

\subsection{主玩家机制}

\term{主玩家}是指\texttt{Player.PLAYER\_ID = 1}的玩家。相对地,\texttt{Player.PLAYER\_ID = 2}的玩家称为\term{副玩家}。
\begin{itemize}
    \item \texttt{Player.PLAYER\_ID} 这个标签在游戏开始前就已经决定,和先后手无关。
    \item 在官方公告\nameref{rule-update:9.2}中有着Player 1与Player 2的称呼,这似乎是承认了之前的主玩家机制并非一个bug。但目前此机制已被移除。
    \item 在冒险模式下,玩家一方始终为主玩家。
    \item 在早期版本和某个未知版本之后,友谊赛的邀请方永远为主玩家。
\end{itemize}

\version{}{17.4.1}\term{主玩家机制}指当由不同玩家操控,或区域不同的扳机同时响应一个事件时,他们的触发顺序不再由入场顺序决定,而是有着一个固定的顺序,依次为:主玩家场上、主玩家手牌、主玩家牌库、副玩家场上、副玩家手牌、副玩家牌库。
\example 你操控\card{格鲁尔},对手操控\card{克尔苏加德}。在回合结束时,主玩家的随从的扳机先触发,无论这是谁的回合以及两个随从哪个先入场。
\notice 主玩家机制只影响扳机顺序,不影响效果顺序,如:AOE的伤害顺序或死亡阶段各死亡事件的顺序等。
\example 双方各操控\card{瑟玛普拉格}并对换\card{战利品贮藏者}。你的战利品先入场并先结算其死亡步骤,你的战利品亡语和对方的瑟玛普拉格按主玩家顺序结算。然后结算对方贮藏者的死亡步骤,对方战利品和你的瑟玛普拉格按主玩家顺序结算。

在当前的游戏中,主玩家机制已被移除,但依然存在主玩家/副玩家之分。主玩家与副玩家的唯一区别就是初始英雄入场顺序,主玩家早于副玩家。这在某些特定情况下会对游戏结算造成影响。
\example 双方均操控\card{扭曲巨龙泽拉库}且未替换过英雄,你使用\card{翡翠掠夺者}。翡翠掠夺者首先对先入场的英雄(即主玩家的英雄)造成伤害,因此主玩家一方先召唤6/6的龙。

\section{效果条件}
\label{effect-cond}

效果条件指战吼和法术的指向和生效都需要满足一定条件。此外,部分特殊事件也有特殊的条件。

战吼与法术条件又分为\term{结算条件}和\term{指向条件}。部分效果内部还有一定的判断条件。

结算条件与效果的内容一般无关,仅是效果执行所必须满足的外部条件。如「如果你持有龙牌」。\\
部分非指向性战吼需要满足一定的结算条件才能执行。这些条件仅在结算时检测——符合即执行,反之不执行。
\example 对手操控四个随从,你使用\card{布莱恩·铜须}和\card{精神控制技师}。第一次战吼随机获得一个敌方随从的控制权,第二次战吼什么也不做。

部分指向性战吼需要满足一定的结算条件才能执行。这些条件在你从手牌中使用时和结算时均检测——都符合即执行。如果结算时符合但使用时不符合,效果不会执行(因为你当时没有为其选择目标)。如果使用时符合但结算时不符合,效果也不会执行(UI会播放事件执行的动画,但实际上没有任何事情发生)。这类效果包括:「如果你持有龙牌」类、\card{麦迪文的男仆}、\card{阴燃电鳗}、\card{火焰使者}、\card{火焰使者}、\card{塔达拉姆王子}、\card{疾疫使者}和\nameref{combo}等。这个规则最先在「如果你持有龙牌」类指向性战吼上发现,因此也称作\term{龙吼二次检测}。
\notice 连击的条件是「本回合使用卡牌数」标签不小于2。对你而言,你的此标签仅在对方回合结束和你的回合开始之间重置,在你的回合结束和对方回合开始之间不重置。
\notice 你可能认为使用时「上个回合使用过元素」而结算时「上个回合未使用过元素」是不可能的。实际上这是通过伊利丹·怒风等扳机将随从的操控权交给对手实现的。另一种实现的方法是在两次战吼之间把随从的操控权交给对手。如以伤害性战吼指向敌方\card{苦痛侍僧},抽到\card{希尔瓦娜斯·风行者}且被\card{人偶大师多里安}复制,再由\card{灵魂歌者安布拉}触发亡语。
\exception \card{穿刺者戈莫克}仅要求使用时符合条件。这是唯一的例外。

部分法术需要满足一定的结算条件才能从手牌中使用,如\card{绝命乱斗}要求场上有两个或更多随从。这些条件仅在你从手牌中使用时检测,如果你使用的法术在结算时不再符合条件,它不会执行任何效果。
\example 对手操控两个随从,你使用\card{绝命乱斗}。结算时如果对手场上只有一个随从,绝命乱斗不会生效。
\example 场上没有其他随从,你使用\card{尤格-萨隆}施放绝命乱斗。由于不符合条件,绝命乱斗不会生效。
\example 场上没有其他随从且你的牌库只有绝命乱斗一种法术,你的\card{资深档案管理员}在回合结束不会施放任何法术。

\term{指向条件}与效果的内容有关,它要求指向性效果必须选择符合这一条件的目标。如「选择一个友方随从」。\\
你在为指向性战吼与法术选择目标时需要满足一定的\term{指向条件}。这些条件仅在你从手牌中使用时检测,如果结算时目标不再满足条件,战吼/法术依然生效。而当战吼被\card{沙德沃克}重复/法术被施放时,只会选择符合条件的目标,如果没有则不执行任何效果。
\example 你持有龙牌并使用\card{燃棘枪兵}指定了一个受伤的敌方随从。如果结算时目标不再受伤,或目标变成了一个友方随从,战吼依然会消灭它。如果结算时你不再持有龙牌,或枪兵被对手操控且对手不持有龙牌,则战吼不执行。
\example 你对你的野兽使用\card{凶猛狂暴},触发敌方\card{扰咒术}。将三个4/6的\card{扰咒师}洗入你的牌库。

部分效果结算过程中会判断一定条件,以确定是否执行某一事件/执行哪一事件。如「对一个随从造成2点伤害,如果该随从是友方恶魔,则改为使其获得+2/+2」。
\example 在\card{高弗雷勋爵}的战吼中,每次 AOE 后都检测「是否有随从死亡」以决定是否执行下一次 AOE。
\example 你对一个友方恶魔使用\card{恶魔之火}使其+2/+2,你的\card{西风灯神}触发并受到2点伤害。

游戏中的部分操作要求目标随从必须在场且非濒死。已知的例子有\nameref{adapt}和「触发一个随从的亡语」。
\example 你使用\card{饥饿的翼手龙}消灭友方\card{阿诺玛鲁斯},随后阿诺玛鲁斯亡语炸死翼手龙。翼手龙死亡,不会弹出进化选项供你选择。
\example 你对友方\card{阿努巴拉克}使用\card{死金药剂}。第一次亡语阿努巴拉克回手并召唤\card[AT-036t]{蛛魔}。第二次亡语不触发,也不会召唤蛛魔。
\exception \card{九命兽魂}可以在手牌触发随从的亡语。

\section{扳机条件}
\label{trigger-cond}

各种扳机的触发都需要满足一定的扳机条件,包括预检测条件、列队条件和结算条件。就目前而言,准确的判明所有扳机的所有条件几乎是不可能的,因此本节只会列出部分受条件影响很大的扳机(如攻击前步骤中的各个扳机),而其他扳机则会给出一个总体上的结论。

\term{预检测条件}指阶段所包含的步骤中的扳机必须在此步骤开始前的一个更早的时间点满足一定条件(也称为通过\term{预检测})才能触发。固有阶段的各个步骤中的扳机在序列开始时进行预检测;死亡阶段各个死亡事件的扳机在死亡检索步骤开始前进行预检测。
\example 对手操控\card{疯狂的科学家},你使用\card{火元素}的战吼杀死科学家并拉出\card{镜像实体}。在使用后步骤中,镜像实体不会复制火元素,因为它在序列开始时不在场。
\example 你使用\card{空中悍匪}的战吼将一张\card{空降歹徒}置入手牌,在使用后步骤中,空降歹徒不会被召唤到战场。因为它在序列开始时不在手牌。
\example 对手使用\card{烈焰风暴}消灭你依次入场的\card{疯狂的科学家}和其他随从。科学家拉出\card{复制}。在随后任何一个随从的死亡步骤中复制都不会触发,因为它在死亡阶段开始时不在场。
\example 你满场且操控\card{轮回}。对手消灭你的一个随从,轮回不能触发。尽管这可能是一个bug,但证实了死亡扳机的预检测在死亡检索步骤开始前。
\notice 只有阶段所包含的步骤中的扳机和死亡事件的扳机会进行预检测,如果一个扳机只是响应一个非死亡事件,如伤害、治疗、召唤、抽牌等,则不需要预检测。
\example 你使用\card{欧克哈特大师},其战吼召唤的三个随从中,2 攻的随从是\card{飞刀杂耍者}。接下来,它可以为3攻随从的召唤而射出一刀。但在完成阶段欧克哈特大师的召唤后步骤中,它无法为欧克哈特大师的召唤而射出一刀。

\term{列队条件}指扳机在它所响应的事件开始结算时需要满足的条件。
\notice 在绝大多数步骤中,多数扳机的列队条件与预检测条件完全相同,且设计一种情况使扳机在预检测和触发时符合条件而列队时不符合是很困难的。因此可以认为列队条件对这些扳机的触发影响不大。
\notice 在攻击前步骤中,多数扳机没有预检测条件(或者说预检测条件仅要求它们在场)。这样它们能在防御者改变后的额外攻击前步骤中触发。
\example 对手操控先入场的\card{游荡怪物}和\card{误导},你操控后入场的\card{诺格弗格市长}。你令市长攻击敌方英雄,触发游荡和误导后仍然攻击敌方英雄,接下来你的市长不触发。因为市长的扳机在列队时不满足「场上有两个或更多的敌方角色」。
\example 与上个例子相似,但触发游荡怪物和误导后转为攻击你的英雄。由于防御者最终发生了改变,产生了额外攻击前步骤。接下来在以你的英雄为目标的额外攻击前步骤中,由于「场上有两个或更多的敌方角色」,你的市长触发并把防御者重新改为对方英雄或游荡怪物新召唤的随从。
\notice 在以上两个例子中,市长的触发条件为「场上有两个或更多的敌方角色」。如果该条件是预检测条件,则在后一个例子中市长不可能触发(因为预检测时显然只有一个敌方角色);如果该条件仅是结算条件,则市长在前一个例子中应该可以触发(因为攻击前步骤中轮到市长时已经有两个敌方角色了)。因此该条件必定是一个列队条件。

\term{结算条件}指扳机在结算时必须满足才能结算的条件。
\notice 如果结算条件不满足,那扳机的闪电符号根本不会亮起,奥秘也不会揭示。这与扳机结算中的分支情况不同。
\notice 一些扳机或明确或隐含地要求随从不处于濒死状态才能触发。其中明确的如\card{恐怖的奴隶主}、\card{腐面}、\card{发条强盗机器人}等;隐含的包括且不限于\card{达利乌斯·克罗雷}、\card{勇敢的记者}、\card{地穴领主}、\card{波戈蒙斯塔}。
\example 你令你的达利乌斯·克罗雷攻击敌方\card{食人魔法师},克罗雷不能触发以获得+2/+2活下来。
\example \version{7.1?}{}对手操控生命值为1的记者。你对其使用\card{死亡缠绕}。记者受到1点伤害,你抽一张牌,但是记者不触发。
\example \version{11.0}{}你使用\card{冰冷触摸}指向你生命值为1的地穴领主。地穴领主受到1点伤害,你召唤一个水元素,但是地穴领主不触发。

对于绝大多数扳机而言,它们的列队条件和结算条件是相同的。但是对非攻击前步骤的、无需预检测的扳机而言,「列队时条件不满足」而「结算时条件满足」的情况是很少见的。对于需要预检测的扳机而言,「预检测时条件满足」、「列队时条件不满足」而「结算时条件满足」的情况也是很少见的。因此,我们很少讨论非攻击前步骤扳机的列队条件。以下是这种情况的一个例子:
\example 你操控先入场的\card{人偶大师多里安}并使用\card{死亡缠绕}指向对手后入场的1血\card{勇敢的记者}。死亡缠绕抽到\card{舞动之剑},人偶大师召唤复制并触发你的\card{灵魂歌者安布拉}。对手抽到一张\card{恩佐斯的子嗣}。对手的人偶大师和灵魂歌者安布拉触发并使濒死的勇敢的记者变为 4/1救了回来。尽管勇敢的记者当前不处于濒死状态,但它在列队时处于濒死状态,因此其扳机无法触发。
\example 对手操控\card{镜像实体}和\card{寒冰克隆}并持有8张手牌。你使用\card{寒光智者}。对手镜像实体触发召唤一个寒光智者的复制,然后对手\card{飞刀杂耍者}射中你的\card{小鬼首领}召唤一个小鬼。你的飞刀杂耍者触发射中对手的\card{咆哮魔}。现在对手有9张手牌。但寒冰克隆不触发,因为在使用后步骤开始时你的手牌是满的。

绝大多数奥秘在无法产生任何效果,或可能起到副作用时,会保持隐藏。
\example 你持有十张手牌且操控\card{寒冰克隆}。对手使用一张随从牌,你的寒冰克隆不会触发。
\example 你手中没有随从牌且操控\card{军备宝箱}。对手使用一张随从牌,你的军备宝箱不会触发。
\example 对手令他的随从攻击你的\card{提里奥·弗丁}。弗丁已经具有圣盾,你的\card{自动防御矩阵}不会触发。
\example 你操控\card{狙击}和\card{忏悔}。对手使用\card{小精灵},狙击对小精灵造成致命伤,而忏悔不触发。
\exception \card{隐秘的智慧}在你拥有10张手牌时依然会触发。这可能是一个bug。

\subsection{攻击前步骤扳机条件}

攻击前步骤中的扳机大多具有较复杂的条件,且其触发与否对战斗的最终结果有较大影响。本部分详细说明这些条件:
\notice 部分条件在卡牌描述中被明确指出(如\card{冰冻陷阱}要求「攻击者是一个随从」),此处不再重复。

\begin{itemize}
    \item \card{冰冻陷阱}的条件为「攻击者在场且未濒死」。
    \item \card{误导}的条件为「攻击者在场且未濒死」和「除攻击者和防御者外,场上至少有一个未濒死,且非免疫的其他角色」。
    \item \card{爆炸陷阱}没有任何条件。
    \item \card{毒蛇陷阱}的条件为「场地未满」。
    \item \card{游荡怪物}的条件为「场地未满」。
    \item \card{崇高牺牲}的条件为「场地未满」。
    \item \card{自动防御矩阵}的条件为「场地未满」和「防御者没有圣盾」。
    \item \card{蒸发}的结算条件为「攻击者在场且未濒死」。
    \item \card{寒冰护体}没有任何条件。
    \item \card{裂魂残像}的结算条件为「场地未满」。
    \item \card{叛变}的条件为「攻击者在场且未濒死」和「攻击者两侧至少有一个未濒死,且非免疫的随从」。
    \item \nameref{forgetful}、\card{食人魔勇士穆戈尔}和\card{诺格弗格市长}的条件为「场上有两个或更多的敌方角色」。
\end{itemize}

\section{随机目标选取条件}

部分效果和扳机需要随机选取目标生效,它们在选择目标时也要满足一定的目标选取条件。

随机伤害效果和扳机会忽略濒死角色,但不忽略免疫角色。
\example 你操控七个\card{炎魔之王拉格纳罗斯},对手空场且具有30点生命。你的回合结束阶段,前四个大螺丝会轰击敌方英雄,而后三个大螺丝不会产生效果。
\example 你操控\card{炎魔之王拉格纳罗斯}而对手操控\card{玛尔加尼斯}。你的回合结束阶段,大螺丝可能会轰击具有免疫的敌方英雄。
\example 你操控一个具有\nameref{poisonous}的\card{飞刀杂耍者}而对手操控三个\card{冰风雪人}。你使用\card{作战动员},不会有任何一个雪人受到两次飞刀。

随机治疗效果和扳机会忽略未受伤角色,但不忽略免疫角色。
\example 你操控七个未受伤的随从且具有18点生命。你使用\card{治疗之雨},所有治疗全部会分配给你的英雄。
\example 你操控\card{光耀之主拉格纳罗斯}且所有友方角色均未受伤。你的回合结束阶段,光螺丝不会触发它的扳机。

随机选择防御者的效果忽略濒死角色和免疫角色。
\example 场上只有一个你操控的\card{玛尔加尼斯}。你令玛尔加尼斯攻击敌方英雄,对手的\card{误导}不触发。

随机增益效果不会忽略濒死角色。
\example 对手操控两个\card{小精灵}和\card{复仇}。你令你后入场的\card{自爆绵羊}攻击一个敌方小精灵,其亡语对另一个小精灵造成致命伤害。此时\card{复仇}触发,濒死的小精灵受到增益效果变为4/1存活。

此外的随机效果一般都会忽略濒死角色。这包括且不限于\card{希尔瓦娜斯·风行者}的随机获得控制权效果、\card{阿努巴尔伏击者}的随机回手效果、\card{狂奔科多兽}和\card{虚空碾压者}的随机摧毁效果、\card{误导}和\card{叛变}的随机选择防御者效果。

还有许多法术牌的效果是忽略濒死角色的。这可以在\card{伊莱克特拉·风潮}生效时,或回合结束时\card{资深档案管理员}连续释放多个法术时体现出来。目前测试过的有:
\begin{itemize}
    \item \card{麦迪文的残影}和\card{幻觉药水}不会给你濒死随从的复制。
    \item 濒死随从不会获得\card{绝命乱斗}的胜利。(需要进一步测试)
\end{itemize}
