\chapter{玩家操作}
\label{player-action}

\term{玩家操作}是玩家在游戏中可以进行的操作。这包括使用一张牌、使用英雄技能、主动攻击、结束回合和投降。

\section{回合结构}
炉石传说的回合分为六个部分,这在游戏中由\texttt{Game.STEP}记录。以你的一个回合为例:
\begin{enumerate}
    \item \term{准备步骤}\texttt{MAIN\_READY}:所有在场实体的在场回合数\texttt{NUM\_TURNS\_IN\_PLAY}加1。恢复你的法力值。将所有「本回合」相关标签归零,如\card{火山幼龙}采用的「本回合死亡随从数」\texttt{Game.NUM\_MINIONS\_KILLED\_THIS\_TURN}、用于确定连击是否触发,以及\card{艾德温·范克里夫}、\card{捕鼠陷阱}等卡牌的效果的「你本回合使用卡牌数」\texttt{Player.NUM\_\allowbreak{}CARDS\_PLAYED\_THIS\_TURN}等。将所有实体的「已攻击次数」标签归零。你的武器变为出鞘状态;对手武器变为入鞘状态。你的奥秘变为不可触发状态;对手奥秘变为可触发状态。
    \item \term{回合开始阶段}\texttt{MAIN\_START\_TRIGGERS}:结算回合开始扳机,包括\card{腐蚀术}、\card{噩梦}等添加的状态、\card{百变泽鲁斯}等卡牌的变形、\card{争强好胜}和\card{死亡暗影}等。
    \item \term{抽牌阶段}\texttt{MAIN\_START}:你抽一张牌。
    \item \term{主时段}\texttt{MAIN\_ACTION}:你可以进行玩家操作,直到选择结束回合或被效果结束回合为止。
    \item \term{回合结束阶段}\texttt{MAIN\_END}:结算回合结束扳机。
    \item \term{清除步骤}\texttt{MAIN\_CLEANUP}:清除所有随从的召唤失调。移除过期的状态,例如\card{叫嚣的中士}或\card{嗜血}所添加的额外攻击力。
\end{enumerate}

\notice 召唤失调是万智牌术语,指随从进入战场的回合不能攻击。在游戏中通过标签\texttt{.JUST\_PLAYED}来记录。
\notice 回合开始、抽牌和回合结束这三个阶段事实上是只包含一个固有阶段的序列。在这三个阶段及之后的死亡阶段(如果有)结束后,会进行胜负裁定。
\example 回合开始时双方都只有1点生命,你的回合开始时\card{鲜血女巫}触发并杀死了你的英雄。游戏结束,你输掉了比赛。你来不及抽到牌库顶的\card{烈焰巨兽}来达成平局。

在两个回合之间也有一个步骤:

\begin{enumerate}
    \item[0.] \term{回合间步骤}\texttt{MAIN\_NEXT}:你的所有手牌入手回合数\texttt{.NUM\_TURNS\_IN\_HAND}加1。若此时没有\card{时空扭曲}等额外回合效果影响,则「当前回合玩家」\texttt{.CURRENT\_PLAYER}切换为你的对手。「游戏总回合数」\texttt{Game.TURN}加1。若游戏总回合数达到90,则游戏立即以平局结束。这意味着一场游戏的回合数上限为89回合。
\end{enumerate}

在酒馆战棋模式中,招募回合和战斗回合将加在一起计算回合数。也就是说,对局将在玩家进行了45次招募和44次战斗之后结束。

\section{使用随从牌}

使用随从牌的序列如下:
\begin{enumerate}
    \item \term{使用阶段}:
    \begin{enumerate}
        \item 你支付费用。\card{海魔钉刺者}等效果在此生效。
        \item 随从进入战场。其区域和位置被设置为对应的值。\texttt{.EXHAUSTED}和\texttt{.JUST\_\allowbreak{}PLAYED}被设置为1。其他相关标签如「本回合使用卡牌数」等进行变动。
        \item 抉择变形和休眠结算。费用状态移除。
        \item \term{使用时步骤}:使用时扳机,例如\card{伊利丹·怒风}、\card{任务达人}、\card{魔能机甲}、\card{无羁元素}、\card{大胖}、\card{人气选手}等列队结算。
        \item \term{召唤时步骤}:召唤时扳机,例如\card{鱼人招潮者}、\card{夜色镇执法官}、\card{饥饿的秃鹫}等列队结算。
        \item 你获得过载。
    \end{enumerate}

    \item \term{结算阶段}:
    \begin{enumerate}
        \item 确定战吼/连击将要触发的次数,\card{布莱恩·铜须}、\card{鲨鱼之灵}、\card{低语元素}的战吼等效果如果满足条件会将下列步骤重复多次。即使这些随从在战吼/连击过程中离开战场,也不会影响这个过程。
        \item 如果随从效果具有目标,指向扳机\card{诺格弗格市长}结算。
        \item 结算战吼/连击/抉择/磁力。
            \exception 抉择变形类随从的抉择由于已结算过,不再处理。
    \end{enumerate}

    \item \term{完成阶段}:
    \begin{enumerate}
        \item \term{召唤后步骤}:召唤后扳机,例如\card{船载火炮}、 \card{飞刀杂耍者}、 \card{公正之剑}、\card{腐化灰熊}、\card{夜色镇议员}等列队结算。
        \item 如果这张牌在序列开始时具有回响,结算回响。
        \item \term{使用后步骤}:使用后扳机,例如腐蚀、\card{镜像实体}、 \card{忏悔}、\card{狙击}、\card{审判}、 \card{变形药水}、\card{顽石元素}、\card{海盗帕奇斯}、\card{低语元素}状态移除、\card{虚空形态}的刷新效果等列队结算。
    \end{enumerate}

\end{enumerate}

如果你仅仅是召唤一个随从(例如\card{战斗号角}、\card{砰砰博士}的战吼、\card{腐面}的受伤扳机等),会进行如下较为简单的流程:
\begin{enumerate}
    \item 移动或创建该实体。休眠结算。
    \item \term{召唤时步骤}
    \item \term{召唤后步骤}
\end{enumerate}

\subsection{随从数量超出上限}

「随从进入战场」步骤不检查随从数量上限。由于支付费用在随从入场之前发生,可以利用这个效果超出场上的随从上限。

\example  你操控四个随从和\card{染病的兀鹫}。你使用\card{海魔钉刺者}和\card{老瞎眼}。首先你受到4点伤害,兀鹫扳机触发召唤一个随从将场上填满。然后老瞎眼入场。此时你操控八个随从。

\subsection{随从在使用过程中离场}

如果你使用的随从牌,或是战吼/连击/抉择指定的目标在结算阶段前离场,其战吼/连击/抉择不会取消,而是以随从的新位置和新数据结算。需要注意,随从进入墓地时会清除伤害和所有状态。
\example 你使用\card{沃金}指定的目标在战吼前被移除,仍然会发生血量交换——因为墓地中的随从数据被重置,你可以获得一个等于目标初始血量的沃金,而不是0血。与此类似,如果沃金在战吼前被移除,它战吼指定的目标会变为2血而不是0血。
\example 你使用的\card{阿古斯防御者}在战吼前被移除,他不会给任何随从嘲讽。详见\nameref{zone}。

如果你使用的随从牌在完成阶段前离场(包括死亡、回手、休眠、磁力、\card{加拉克苏斯大王}的替换英雄等),完成阶段内的多数扳机都不会触发。也就是说,这些扳机都要求随从在场。
\example 你操控\card{布莱恩·铜须}并使用\card{负伤剑圣},剑圣的战吼杀死自己。随后的完成阶段,你的\card{飞刀杂耍者}不会触发,对手的\card{镜像实体}也不会触发。
\exception 部分与随从相关的使用后扳机仍对离场随从生效。这包括\card{全息术士}和\card{伊克斯里德,真菌之王}。
\exception 休眠随从可以触发\card{飞刀杂耍者}、\card{下水道渔人}等卡,但无法触发\card{火焰术士弗洛格尔}。
\exception 部分完成阶段扳机只与「你使用牌这个动作」或「你使用的那张牌」相关,而非「你使用的随从」。它们在随从离场的情况下仍然触发。这包括:\card{军备宝箱}、\card{捕鼠陷阱}和\card{隐秘的智慧}、\card{寒冰克隆}、\card{湿地女王}、\card{探索地下洞穴}、\card{虚空形态}的刷新效果和\card{格林达·鸦羽}。

如果你使用的随从在完成阶段前离场而又进场,完成阶段的扳机可以触发。
\example 你操控\card{飞刀杂耍者}并使用\card{恐惧地狱火}战吼杀死\card{阿努巴尔伏击者}和\card{空灵召唤者}。伏击者将地狱火弹回手牌,空灵再将地狱火召唤出来,飞刀因此触发一次。完成阶段由于地狱火在场,飞刀再触发一次。
\example 你使用\card{唤魔者克鲁尔}。\card{伊利丹·怒风}、\card{飞刀杂耍者}、\card{阿努巴尔伏击者}将克鲁尔在战吼前弹回手牌。克鲁尔战吼再将自己召唤出来,飞刀因此触发一次。完成阶段由于克鲁尔在场,飞刀再触发一次。

\subsection{随从在使用过程中被变形}

如果随从在使用过程中被变形,所有扳机仍然正常触发。

召唤「你使用的随从的复制」的牌,召唤的是那个变形后随从的复制。\card{软泥教授弗洛普}给的也是变形后的牌。
\example 你使用\card{无面操纵者}指定一个随从为目标。敌方\card{镜像实体}、\card{全息术士}和你的\card{伊克斯里德,真菌之王}召唤的都是变形之后的随从,而不是无面本体。
\example 你使用\card{血沼迅猛龙}。敌方\card{变形药水}和镜像实体依次触发,对手得到一个绵羊。
\example 你操控后入场的真菌之王,对手操控先入场的变性药水。你使用血沼迅猛龙。首先敌方变形药水触发,然后你的真菌之王触发,你场上现在有两个绵羊。如果你的真菌之王先入场,那么它先触发,最后你场上有一个绵羊和一个迅猛龙。

\card{湿地女王}的触发条件是:你使用这个随从时该随从为1费,且该随从没有经过变形。
\example 你使用1费的\card{无面操纵者}复制\card{石牙野猪}。湿地女王不触发。
\example 对手先使用\card{变形药水},然后你使用湿地女王和石牙野猪。敌方\card{变形药水}触发,湿地女王不触发。相反,如果你先使用湿地女王,它会在变形药水之前触发。
\notice 即使变形是将该随从变成相同的随从,湿地女王仍不能触发。
\example 将上个例子的石牙野猪换成\card[CS2_tk1]{绵羊}。湿地女王仍然不触发。

\card{探索地下洞穴}的计数所考虑的是变形之前的那个随从。
\example 你使用探索地下洞穴,然后使用\card{无面操纵者}复制\card{血沼迅猛龙},最后使用无面操纵者复制\card{淡水鳄}。探索地下洞穴的进度为2,且记录的是无面操纵者。

\nameref{echo}在\card{变形药水}之前触发,给你的是变形之前的那个随从。

\card{顽石元素}是否触发根据变形前的随从而定。\card{海盗帕奇斯}、\card{空降歹徒}、\card{小型法术红宝石}和\card{奥术守护者卡雷苟斯}则需要变形前后(预检测时与结算时)都满足条件。
\example 你操控一个\card{顽石元素}。你使用\card{无面操纵者}复制非战吼随从。顽石元素触发。
\example 你牌库中有一张\card{海盗帕奇斯}。你使用\card{无面操纵者}复制海盗。帕奇斯不触发。
\example 你牌库中有一张\card{海盗帕奇斯}。你使用\card{南海船工},然后触发敌方\card{变形药水}。帕奇斯不触发。
\example 你操控一个\card{奥术守护者卡雷苟斯}。你使用\card{无面酒客}复制非战吼随从。奥术守护者卡雷苟斯不触发。但如果复制战吼随从。奥术守护者卡雷苟斯触发。

\subsection{随从在使用过程中转移控制权}

如果你使用的随从牌在结算阶段前被对手获得了控制权,其战吼/连击/抉择依然正常结算。
\example 你使用\card{夜刃刺客}但结算阶段前它被敌方\card{希尔瓦娜斯·风行者}的亡语夺取了控制权。结算阶段,夜刃刺客的控制权为对手所有,其战吼对你的英雄造成3点伤害。

如果你使用的随从牌在结算阶段中被对手获得了控制权,其战吼/连击/抉择依然正常结算。
\notice 这一般是通过分为多个步骤执行的战吼,或\card{布莱恩·铜须}等效果来实现。
\example 你操控布莱恩·铜须并使用\card{希拉斯·暗月},选择将暗月交给对手。第二次战吼结算时暗月由对手操控,所以随机选择一个方向。
\example 你使用\card{寒光智者}。其战吼首先为它的控制者,即你抽两张牌。你抽到\card{惊奇卡牌}释放\card{变节}将寒光智者交给了对手。接下来,其战吼为它控制者的对手,即你抽两张牌。

如果你使用的随从牌在完成阶段前被对手获得了控制权,完成阶段内的绝大多数扳机不会触发。
\exception \card{捕鼠陷阱}和\card{隐秘的智慧}依然能够触发。上文中提到的其他能在随从离场时触发的扳机均不能在随从控制权转移的情况下触发。

\section{使用法术牌}

\notice 在游戏中存在着两种不同的\term{施放}。\card{狂野炎术师}的「施放」指「从手牌中使用」;而\card{惊奇套牌}的「施放」指「执行法术效果」。在本规则集中为避免混淆,将「从手牌中使用」称为「使用」,而「执行法术效果」称为「施放」。
\exception \nameref{cast-when-drawn}是一个抽牌扳机。它并不是施放或使用这个法术。
\exception \card{亵渎}、\card{传播瘟疫}、\card{命运骨骰}和\card{永恒之火}等牌中的「施放」实为「重复法术效果」(包括过载)。
\exception \card{白衣幽魂}并不是真的施放了很多\card{心灵之火}。
\exception \card{相位追猎者}和\card{套圈圈}等施放奥秘的牌只是将奥秘置入战场(它并不会消耗风潮的光环)。

使用法术牌的序列如下:
\begin{enumerate}
    \item \term{使用阶段}:
    \begin{enumerate}
        \item 你支付费用。
        \item 如果对手操控\card{法术反制},则阻止这个法术生效。序列中的绝大多数步骤都会被取消,详见\nameref{counter}。
        \item 如果对手操控\card{古神在上},则将此法术替换为另一个费用相同的法术。序列中的绝大多数步骤都会为这个新法术而结算。
        \item 费用状态移除。已测试的包括\card{血色绽放}、\card{古加尔}、\card{肯瑞托法师}、\card{暗金教侍从}、\card{墨水大师索莉娅}、\card{伺机待发}和\card{亡鬼幻象}。此外,\card{暮陨者艾维娜}的光环也在此切换。
        \item 如果这张法术是奥秘或任务,它进入奥秘区;否则它进入战场。其他相关标签如「本回合使用卡牌数」等进行变动。
        \item \term{使用时步骤}:其它使用时扳机,例如\card{伊利丹·怒风}、\card{任务达人}、\card{魔能机甲}、\card{无羁元素}、\card{紫罗兰教师}、\card{法力浮龙}、\card{青蛙之灵}等列队结算。
        \item 你获得过载和双生法术牌。
    \end{enumerate}

    \item \term{结算阶段:}
    \begin{enumerate}
        \item 确定法术执行的次数,\card{伊莱克特拉·风潮}的战吼等效果如果满足条件会将法术描述重复两次,且你会再次获得过载。即使风潮的效果在法术执行过程中失去,也不会影响这个过程。
        \item 如果法术具有目标,指向扳机\card{诺格弗格市长}和\card{扰咒术}列队结算。
        \item 结算法术描述。如果是一个奥秘或任务,没有事情发生。
        \item 如果该法术不是奥秘或任务,它进入墓地。
    \end{enumerate}

    \item \term{完成阶段}:
    \begin{enumerate}
        \item 如果这张牌在序列开始时具有回响,结算回响。
        \item \term{使用后步骤}:使用后扳机,例如\card{法术共鸣}、\card{狂野炎术师}、 \card{火妖}、\card{西风灯神}、\card{普崔塞德教授}、\card{虚空形态}的刷新效果、\card{星界密使}状态移除、\card{伊莱克特拉·风潮}状态移除等列队结算。
    \end{enumerate}
\end{enumerate}

如果你仅仅是施放一个法术(例如\card{尤格-萨隆}、\card{惊奇卡牌}、\card{资深档案管理员}等),会进行如下较为简单的流程:
\begin{enumerate}
    \item 如果该法术是奥秘或任务,它进入奥秘区(无效的奥秘会被展示并送入墓地);否则它进入战场。
    \item 你获得过载和双生法术牌。
    \item 结算法术描述。施放的法术可以享受法术伤害加成。
    \item 如果使用的法术不是奥秘或任务,它进入墓地。
    \item \term{使用后步骤}。这个使用后步骤包括\card{星界密使}与\card{伊莱克特拉·风潮}等卡的状态移除。
\end{enumerate}

\card{西风灯神}和\card{沃拉斯}的效果并不是施放法术。它们的流程仅包括:
\begin{enumerate}
    \item 你获得过载和双生法术牌。
    \item 结算法术描述。西风灯神和沃拉斯的效果可以享受法术伤害加成。
\end{enumerate}


\subsection{目标在使用过程中离场}

与战吼类似,如果你法术指定的目标在结算前离场,法术不会取消,而是以目标的新位置和新数据结算。
\example 你对敌方\card{暮光幼龙}使用\card{死亡缠绕},在死缠结算前暮光龙被弹回手牌。接下来死缠结算时,首先将手牌中的4/1暮光龙打至4/0,然后判定其受致命伤并抽一张牌。注意在手牌中受致命伤的随从并不会真正死去,这是因为死亡检索只会考虑在场的角色。

\subsection{特殊的法术互动}

「某法术再结算一次」指在通常的一次结算之后,你先获得这个法术的过载和双生法术牌,再结算一次它的描述。
\example 你使用\card{伊莱克特拉·风潮}再使用\card{闪电箭}。使用阶段你获得过载1,然后在描述阶段你首先对目标造成3点伤害,再获得过载1,再对该目标造成3点伤害。
\notice \card{星界密使}对整个法术的多次结算都有效,然后在法术结算完成后移除法强状态。

\card{伊莱克特拉·风潮}和\card{星界密使}和法术之间有与众不同的互动:它们不仅能与使用的法术产生互动,而且也能与施放的法术产生互动。施放法术的情况下,它们添加的状态在对应的使用后时机移除,也就是在法术效果结算完之后。
\example \version{13.0}{}你使用风潮,然后装备\card{符文之矛}并攻击。符文之矛发现的法术会结算两次。在此之后你使用\card{闪电箭}。闪电箭不会结算两次。
\example \version{12.2}{}你使用星界密使,然后使用\card{尤格-萨隆的仆从}。副导演施放的法术会获得法术伤害 +2。在这之后你使用\card{寒冰箭}。寒冰箭不会获得法术伤害 +2。
\notice 在之前的版本里,这些战吼的状态只会被使用的法术所消耗。也就是说,如果你使用星界密使然后使用\card{尤格-萨隆},导演施放的所有法术都会受到法强加成。

\card{泽蒂摩}修改结算阶段的结算流程。在通常的一次结算之后,该法术将以相邻的随从为目标再结算一次法术效果,先左再右。
\notice 泽蒂摩和\card{伊莱克特拉·风潮}的互动为:将「结算一次,然后对相邻的随从再结算一次」重复两次。
\notice 泽蒂摩的额外结算也能享受\card{星界密使}状态的临时法强。
\notice 泽蒂摩的效果看似是一个扳机(有闪电符号),但是实质上它是一个光环。也就是说,如果第一次法术结算将泽蒂摩沉默或变形,它仍然会再执行两次法术效果;不过这个时候闪电符号就不会闪亮了。
\example 你从左到右操控\card{血沼迅猛龙}、泽蒂摩和\card{淡水鳄},然后对泽蒂摩施放\card{妖术}。虽然泽蒂摩首先被变形,但是迅猛龙和淡水鳄随后也被变形。
\notice 「以相邻的随从为目标再结算一次法术」的效果不受\card{诺格弗格市长}影响。
\notice 泽蒂摩对施放的法术不会生效。

\section{使用武器牌}

使用武器牌的序列如下:
\begin{enumerate}
    \item \term{使用阶段}:
    \begin{enumerate}
        \item 你支付费用。武器进入战场。费用状态移除。
            \notice 此时武器已经可以触发其扳机,例如\card{公正之剑}和\card{诅咒之刃}的扳机。
        \item 其他相关标签如「本回合使用卡牌数」等进行变动。
        \item \term{使用时步骤}:使用时扳机,例如\card{伊利丹·怒风}、\card{任务达人}、\card{魔能机甲}、\card{人气选手}等列队结算。
        \item 你获得过载。
    \end{enumerate}

    \item \term{结算阶段}:
    \begin{enumerate}
        \item 确定战吼将要触发的次数,\card{布莱恩·铜须}、\card{低语元素}的战吼等效果如果满足条件会将下两个步骤重复多次。即使这些随从在战吼过程中离开战场,也不会影响这个过程。
        \item 如果武器效果具有目标,指向扳机\card{诺格弗格市长}结算。
        \item 结算战吼/连击。
        \item 消灭你的旧武器。「你装备的武器」\texttt{Player.334}被设置为新武器。\card{锈水海盗}和\card{小型法术秘银石}在此时结算。
    \end{enumerate}

    \item \term{完成阶段}:
    \begin{enumerate}
        \item \term{使用后步骤}:使用后扳机,例如\card{瑟拉金之种}、\card{捕鼠陷阱}、\card{虚空形态}的刷新效果等列队结算。
    \end{enumerate}
\end{enumerate}
​
如果你仅仅是装备一把武器(例如\card{升级!}、\card{恩佐斯的副官}、\card{匕首精通}等),会进行如下较为简单的流程:
\begin{enumerate}
    \item 消灭你的旧武器。
    \item 新武器进入战场。\card{锈水海盗}和\card{小型法术秘银石}在此时结算。
    \item 「你装备的武器」被设置为新武器。
\end{enumerate}
​
\subsection{同时持有两把武器}

当你使用武器牌时,在结算阶段替换前你仍装备着旧武器。此时你的新武器和旧武器都在场,它们可以互相影响,或触发它们的扳机。
\example 你在持有\card{圣光的正义}时使用\card{真银圣剑},使用阶段通过\card{伊利丹·怒风}等扳机杀死你的\card{提里奥·弗丁}。\card{灰烬使者}替换圣光的正义。在接下来的结算阶段,真银圣剑替换灰烬使者。
\example 你在持有\card{弑君}使用另一把弑君,使用阶段通过伊利丹等扳机杀死你的\card{南海畸变船长}。其亡语会为旧弑君增加2点攻击。
\example 你在装备\card{公正之剑}时使用另一把公正之剑。使用阶段\card{伊利丹·怒风}触发,两把武器均对召唤的\card[EX1_614t]{埃辛诺斯之焰}生效,使其变为4/3。结算阶段你当前武器变为新公正之剑,此时它仅剩4点耐久。
\example 你在装备公正之剑时使用\card{青玉之爪}。结算阶段青玉之爪发动战吼召唤1/1\card[CFM_712_t01]{青玉魔像},公正之剑触发使其变为2/2。
\example 你在装备\card{诅咒之刃}时使用连击生效的\card{毁灭之刃}并指向你的英雄。结算阶段你的英雄会受到4点伤害。
​
\section{使用英雄牌}

使用英雄牌的序列如下:

\begin{enumerate}
    \item \term{使用阶段}:
    \begin{enumerate}
        \item 你支付费用。英雄牌进入战场。费用状态移除。
        \item 其他相关标签如「本回合使用卡牌数」等进行变动。
        \item \term{使用时步骤}:使用时扳机,例如\card{伊利丹·怒风}、\card{任务达人}、\card{魔能机甲}、\card{人气选手}等列队结算。
        \item 新英雄的最大生命值、当前生命值和护甲被设定为和旧英雄相同,并获得英雄牌上标注的额外护甲。
        \item 移除你的旧英雄技能和旧英雄。将「你的英雄」 \texttt{Player.HERO\_ENTITY} 设置为新英雄。
    \end{enumerate}

    \item \term{结算阶段}:
    \begin{enumerate}
        \item 你获得新的英雄技能。
        \item 确定战吼将要触发的次数,\card{布莱恩·铜须}、\card{低语元素}的战吼等效果如果满足条件会将下列步骤重复多次。即使这些随从在战吼过程中离开战场,也不会影响这个过程。
        \item 结算战吼/抉择。
    \end{enumerate}

    \item \term{完成阶段}:
    \begin{enumerate}
        \item \term{使用后步骤}:使用后扳机,例如\card{瑟拉金之种}、\card{捕鼠陷阱}等列队结算。
    \end{enumerate}
\end{enumerate}

每当你替换英雄时(无论是使用英雄牌或者是其它方式),以下事情发生:
\begin{itemize}
    \item 你不再被\nameref{freeze}。
    \item 你的「本回合已经攻击次数」保留。
    \item 你替换英雄技能。(使用英雄牌时的替换技能已经在描述阶段中描述了)
    \item 你的职业变为新英雄的职业。你的种族变为对应种族。
\end{itemize}

如果你替换成的是\card[EX1_23h]{加拉克苏斯大王}或\card[BRM_027h]{炎魔之王拉格纳罗斯}时,还会额外发生:

\begin{itemize}
    \item 移除你英雄具有的所有状态,例如\card{暗影之刃}或\card{寒冰屏障}添加的。
    \item 你的英雄的护甲与受到的伤害归零;你的最大生命值不变。这与使用英雄牌时不同。
\end{itemize}

\notice \card[BRM_027h]{炎魔之王拉格纳罗斯}\emph{不具有}元素种族。

\example 你操控\card{伊利丹·怒风}和\card{无证药剂师}且具有5点生命。你使用一张英雄牌。首先伊利丹触发,导致无证药剂师将你的生命值降为0,然后你才获得护甲值。因此你的英雄被判定为死亡,你输掉了这盘游戏。

\example 你使用\card{暗影收割者安度因}并通过\card{伊利丹·怒风}等扳机杀死一个友方\card{管理者埃克索图斯},使你的英雄在结算阶段前变为\card[BRM_027h]{炎魔之王拉格纳罗斯}。结算阶段,你的英雄技能\card[BRM_027p]{死吧,虫子!}被\card{虚空形态}替代,然后发动暗影收割者安度因的战吼。而你的英雄仍是炎魔之王拉格纳罗斯。

\subsection{异常地替换英雄}
有几种方式可以「异常地」替换英雄。替换后的英雄是初始血量及上限、初始护甲且没有英雄技能的。这说明,替换英雄这个操作本身仅仅包括新英雄入场、移除旧英雄和英雄技能、修改「你的英雄」三部分。其余部分都是特定的结算过程中添加的额外修饰。
\notice 所有的英雄牌本身就是30血5护甲的实体,因此通过这种异常方式替换的英雄也是30血5护甲。
\example \version{}{11.2} 你对敌方\card{变色龙卡米洛斯}使用\card{埋葬},并在法术结算之前消灭了它。随后你抽到了它,它在手牌中变形成英雄牌。你使用\card{复活术}复活唯一死亡的变色龙。结果你异常替换了英雄。参见\href{https://www.bilibili.com/video/av24248527}{av24248527}。
\example 你对你的\card{阿努巴拉克}使用\card{食肉魔块}。阿努巴拉克移回手牌;你使用它,将它变形成\card{变色龙卡米洛斯}再移回手牌。随后变色龙变形为英雄牌。你消灭食肉魔块。结果你异常替换了英雄。参见\href{https://www.bilibili.com/video/av37117483}{av37117483},7:56处。

\section{战斗}
\label{battle}

当一个角色主动攻击时,会产生如下的序列:
\begin{enumerate}
    \item \term{战斗阶段}:
    \begin{enumerate}
        \item \term{攻击前步骤}:攻击前扳机列队结算。例如\card{爆炸陷阱}、\card{冰冻陷阱}、\card{误导}、\card{毒蛇陷阱}、\card{崇高牺牲}、\card{蒸发}、\card{寒冰护体}、\nameref{forgetful}及类似效果、\card{诺格弗格市长}、\card{叛变}、\card{游荡怪物}、\card{自动防御矩阵}等。
        \item 如果结算完毕后防御者发生了改变,则再进行一次攻击前步骤。重复此步骤直到防御者没有发生改变为止。在额外的步骤中,已经在之前的攻击前步骤触发过的攻击前扳机不能再次触发。
        \item \term{攻击时步骤}:攻击时扳机列队结算。例如\card{真银圣剑}、\card{血吼}、\card{智慧祝福}、\card{窃贼}、\card{真言术:耀}、\card{“鲨鱼”加佐}、\card{收集者沙库尔}、\card{失心农夫}等。
        \item 如果此时攻击者、防御者或任意一方英雄离场、濒死或休眠,则攻击取消,跳过伤害和攻击后步骤。
        \item \term{伤害步骤}:攻击者的潜行移除。然后攻击者对防御者造成伤害同时防御者对攻击者造成伤害。最后结算两者的伤害事件。
        \item 无论攻击是否被取消,攻击者的「本回合已攻击次数」\texttt{.NUM\_ATTACKS\_THIS\_\allowbreak{}TURN}加1。
        \item \term{攻击后步骤}:攻击后扳机,例如\card{捕熊陷阱}、\card{腐树巨人}、\card{符文之矛}等列队结算。
            \notice 如果攻击者是英雄且装备武器,武器失去耐久也与这些扳机共同列队结算。
        \item \card{蜡烛弓}和\card{角斗士的长弓}添加的临时免疫移除。若防御者濒死则攻击者的\card{霜之哀伤}将其记录。
    \end{enumerate}
\end{enumerate}

如果一个角色仅仅是受到某个效果(例如\card{沼泽之王爵德}、\card{狗头人蛮兵}、\card{野兽之心}、\card{群体狂乱}、\card{超级对撞器}和\card{决斗})强制进行了一次攻击,会进行如下较为简单的流程:
\begin{enumerate}
    \item  上文中战斗阶段的所有流程。由于此时处于一个阶段中,在此之后没有阶段间步骤。此外,野兽之心、群体狂乱、超级对撞器和决斗等效果不会使「本回合已攻击次数」加1。
\end{enumerate}

\exception \card{诺格弗格市长}和\card{食人魔勇士穆戈尔}对连续的多次强制战斗只会触发一次。这主要表现在\card{疯狂巨龙死亡之翼}上。

由于额外的攻击前步骤的存在,直观上看攻击前扳机的触发顺序与入场顺序可能不同。
\example 对手操控\card{毒蛇陷阱}和\card{游荡怪物},你攻击敌方英雄。攻击前步骤中游荡怪物触发,将防御者改为新召唤的随从。然后在额外攻击前步骤中,毒蛇陷阱触发。

如果在同一个攻击前步骤中防御者发生了改变,该步骤中的后续扳机仍按原防御者进行结算;那之后才进行一个额外的攻击前步骤,在这个额外步骤中扳机按新防御者进行结算。
\example 对手操控\card{误导}和\card{爆炸陷阱},你攻击敌方英雄。尽管误导先将防御者改为了一个随从,爆炸陷阱仍认为对方英雄受到攻击,因此可以触发。
\example 对手操控\card{崇高牺牲}和\card{自动防御矩阵},你攻击敌方\card{小精灵}。尽管崇高牺牲先将防御者更改为2/1的\card{防御者},自动防御矩阵仍认为小精灵受到攻击,因此为小精灵而触发。小精灵会获得圣盾,而2/1的\card{防御者}成为被攻击目标。
\example 对手操控\card{诺格弗格市长}和\card{自动防御矩阵},你攻击敌方英雄。市长先将防御者更改为\card{小精灵},而自动防御矩阵不属于这一攻击前步骤中的扳机(因为受到了攻击的并不是随从)。在接下来的额外攻击前步骤中自动防御矩阵触发,使小精灵获得圣盾。

\nameref{forgetful}及类似效果如果符合条件但因50\%的概率而没能触发,则它也不能在后续的额外攻击前步骤中触发。
\example 双方操控14个\card{食人魔勇士穆戈尔}。对于每一次战斗而言,穆戈尔触发的数量约为平均值7个。倘若此规则不成立,即没能触发的穆戈尔可以在后续额外攻击前步骤中触发,那么每一次战斗穆戈尔触发的数量会接近为14个。

如果在同一个攻击前步骤中,防御者先从A变为B再变为C,则不会产生以B为防御者的额外攻击前事件。
\example 对手操控\card{游荡怪物}、\card{误导}和\card{毒蛇陷阱},你令你的随从攻击敌方英雄。攻击前步骤中首先游荡怪物触发将防御者改为随从,然后误导触发又将防御者改为你的英雄。在这之后,毒蛇陷阱不会触发。

如果在同一个攻击前步骤中,防御者先从A变为B再变为A,则不会产生额外攻击前事件。
\example 对手操控\card{飞刀杂耍者}、\card{叛变}、\card{游荡怪物}和\card{误导},你令你的唯一随从\card{恐怖的奴隶主}攻击敌方英雄。首先叛变不触发;然后游荡怪物触发召唤随从成为防御者,且飞刀射中奴隶主召唤一个新奴隶主;再之后误导触发将防御者改回为敌方英雄。此时不会产生额外攻击前事件,叛变不能触发。

多数攻击前扳机都有比较复杂的扳机条件,这决定了他们与其他攻击前扳机的互动结果。这部分内容会在\nameref{trigger-cond}中详细说明。

\card{蜡烛弓}与\card{角斗士的长弓}除了具有一个攻击前扳机之外,还具有一个预伤害扳机。它的作用是在你真正获得免疫状态之前防止你受到的所有伤害。

如果攻击者在攻击前步骤中濒死,它可以被攻击步骤中的扳机救回从而完成攻击。
\example 你具有1点生命且对手操控\card{爆炸陷阱},你装备\card{真银圣剑}攻击敌方英雄。首先爆炸陷阱对你造成2点伤害,然后真银圣剑使你恢复2点生命。此时你的生命值为正,攻击可以完成。

除了攻击者/防御者离场或濒死,大多数其他情况都不会导致攻击取消。这包括防御者变成友军,攻击者变成敌人,攻击者变为0攻或攻击者的武器被新武器替换等情况。
\example 对手操控\card{爆炸陷阱}和\card{游荡怪物},你装备\card{蜡烛弓}攻击敌方英雄。首先爆炸陷阱对你的\card{苦痛侍僧}造成2点伤害,其抽到\card{提里奥·弗丁}。\card{人偶大师多里安}和\card{灵魂歌者安布拉}触发使你的武器被\card{灰烬使者}替换。然后游荡怪物触发召唤一个随从成为防御者。在接下来的战斗中,你使用灰烬使者对新的防御者造成5点伤害,且你仍受到蜡烛弓的免疫效果保护。

在伤害步骤中,伤害事件会在双方均造成伤害后结算,先结算攻击者对防御者的伤害事件,再结算防御者对攻击者的伤害事件。
\example 你令\card{暴乱狂战士}攻击对手暴乱狂战士。它们各对对方造成2点伤害,然后它们均变为4攻。
\example 对手操控\card{飞刀杂耍者},你令\card{恐怖的奴隶主}攻击敌方\card{小鬼首领}。首先结算防御者小鬼首领的受伤事件,召唤小鬼触发飞刀射中奴隶主。然后结算攻击者奴隶主的受伤事件,由于此时奴隶主濒死其扳机无法触发。

\notice 「攻击者对防御者造成伤害」这一事件包括「攻击者造成伤害」和「防御者受到伤害」。因此响应「攻击者造成伤害」的扳机和响应「防御者受到伤害」的扳机按顺序触发。「防御者对攻击者造成伤害」事件同理。
\example 你具有2点生命,并令\card{兽人铸甲师}攻击敌方\card{掷斧者}。若铸甲师先入场,则它首先触发,你首先获得护甲而存活;若铸甲师后入场,则掷斧者先触发,你最终死亡。

如果伤害步骤后攻击者、防御者或任意一方英雄离场或濒死,攻击后步骤不会取消。
\example 对手具有1点生命且操控\card{捕熊陷阱}和具有\nameref{lifesteal}的\card{飞刀杂耍者},你令\card{小精灵}攻击对手英雄。攻击后步骤中捕熊触发,飞刀射中小精灵并吸血。最后对手恢复到1点生命而存活。

\card{符文之矛}施放法术是在额外阶段中执行的,这可以避免战斗中死亡的随从又被法术救活。
\example 对手操控\card{火羽精灵}和\card{愤怒的小鸡},你使用\card{符文之矛}攻击愤怒的小鸡并施放\card{衰变}。愤怒的小鸡立刻死亡,只有火羽精灵会被衰变。
\example 上例中如果你施放的是\card{妖术}且场上没有其他随从,则妖术必定以火羽精灵为目标。

\card{凶恶的雏龙}的进化很晚才实际生效。
\example 你使用凶恶的雏龙并将它打到一血。对手使用\card{火焰结界},然后你令雏龙攻击敌方英雄。在攻击后步骤中首先雏龙进化,你选择 +3 生命值,但这一效果稍后才会执行。接下来火焰结界触发,雏龙濒死。战斗阶段结束,你的雏龙来不及获得 +3 生命值就已经死亡。
\notice 进化效果不能在目标濒死时生效,详见\nameref{adapt}。
\example 将上个例子的雏龙和火焰结界的顺序交换。在攻击后步骤中首先火焰结界触发,雏龙濒死,因此进化不生效。战斗阶段结束,你的雏龙死亡。
\notice \version{}{8.0}发现效果的额外阶段在其他序列中也存在。
\notice \version{}{?}在之前的版本中\card{符文之矛}施放的法术也是在额外阶段中执行的。
\example \version{}{?}对手操控\card{火羽精灵}和\card{愤怒的小鸡},你使用符文之矛攻击愤怒的小鸡并施放\card{衰变}。愤怒的小鸡立刻死亡,只有火羽精灵会被衰变。
\example \version{}{?}战士使用1耐久的符文之矛攻击并施放\card{升级}。符文之矛首先因耐久度用光而摧毁,你只能获得一把1/3的武器。

\subsection{攻击次数}

判断一个角色能否攻击,是通过比较「本回合已攻击次数」与「本回合可攻击次数」来实现的。如果小于则可攻击。

一名角色每进行一次攻击,它的「本回合已攻击次数」会加1。\card{沼泽之王爵德}和\card{狗头人蛮兵}的效果也会使「本回合已攻击次数」加1。「本回合已攻击次数」会在随从区域移动时,及回合间清除步骤中清零;但不会在随从控制权转移时清零。
\example 你令\card{银色指挥官}攻击敌方英雄,然后依次对其使用\card{变节}和\card{精神控制}。它不能再次攻击。
\example 对手操控一个3攻的\card{沼泽之王爵德},你先使用\card{小精灵}后对王爵德使用\card{暗影狂乱}。它不能攻击。
\notice 并非所有由卡牌效果导致的强制战斗都会使「本回合已攻击次数」加1。目前暂时没有明确的规律。
\example 你对一个本回合可进行攻击的随从使用\card{横冲直撞},它在攻击相邻的随从后,仍然可以再次攻击。
\example 你控制一个本回合可进行攻击的随从,你装备\card{引月长弓},并攻击一个随从。它在攻击目标随从后,仍然可以再次攻击。
\example 你使用\card{疯狂巨龙死亡之翼},它攻击数个随从。战吼后,你对他使用\card{火箭靴},它不能攻击。

「本回合可攻击次数」由\term{基本可攻击次数}和\term{额外可攻击次数}两部分组成。
\begin{itemize}
    \item 一个角色的\term{基本可攻击次数}通常为1,\nameref{windfury}、超级风怒和\card{愚者之灾}分别将目标的基本可攻击次数改为2、4 和无穷。
    \item 一个角色的\term{额外可攻击次数}通常为0,\card{巨型沙虫}、\card{贡克,迅猛龙之神}、\card{鞭笞者苏萨斯}等扳机触发时会使目标的额外可攻击次数加1。
\end{itemize}
\example 你控制\card{贡克,迅猛龙之神},持有2耐久的\card{愚者之灾}并使用\card{英勇打击}攻击\card{摩天龙}两次将其消灭。此时由于愚者之灾被摧毁,你的基本攻击次数为 1;贡克使你的额外攻击次数也变为1。本回合已攻击次数为2,不小于本回合可攻击次数。因此你不能再次攻击。

此外,还有一个角色可以攻击还要满足许多其他条件。如不具有召唤失调、不具有「无法攻击」、攻击力不为0、有合法的攻击目标等。但如果是效果导致的强制战斗,那么一般来说只要有合法的攻击目标即可攻击。
\exception \card{决斗}召唤的无法攻击的随从不能进行攻击。

\subsection{强制战斗}
\label{forced-battle}

一些效果可以让随从进行战斗。这类效果可能消耗也可能不消耗随从的攻击次数;有可能考虑冻结也有可能不考虑。一些结果见下表。

\begin{center}
\begin{tabular}{|c|c|c|}
\hline
卡牌 & 是否消耗攻击次数 & 是否适用于冻结角色 \\
\hline
洛萨 & \cmark & \cmark \\
伊利达雷审判官 & \xmark & \cmark \\
\hline
\end{tabular}
\end{center}

\section{使用英雄技能}

使用英雄技能的序列如下:
\begin{enumerate}
    \item \term{结算阶段}:
    \begin{enumerate}
        \item 如果英雄技能具有目标,指向扳机\card{诺格弗格市长}结算。
        \item 结算英雄技能。「你本回合已使用英雄技能次数」加1。
    \end{enumerate}

    \item \term{激励阶段}:
    \begin{enumerate}
        \item 激励扳机,例如\card{低阶侍从}、\card{暗影子嗣}、\card{毒镖陷阱}、\card{寒冰行者}、\card{大胆的吞火者}的状态移除等列队结算。
    \end{enumerate}
\end{enumerate}

\subsection{特殊的英雄技能互动}

\card{寒冰行者}可以在目标离场的情况下触发。这包括被移除、被\card{角斗场主管}回手等等。
\example 你操控\card{寒冰行者}并使用\card{火焰冲击}杀死你的\card{沟渠潜伏者}。其亡语召唤\card{莫拉比}。在接下来的激励阶段,寒冰行者将墓地中的沟渠潜伏者冻结,莫拉比响应这次冻结并将一张沟渠潜伏者的复制加入你的手牌。

\card{龙鹰之灵}修改使用英雄技能的结算流程。在通常的一次结算之后,该英雄技能将以相邻的随从为目标再结算一次技能效果,先左再右。
\notice 龙鹰之灵和\card{寒冰行者}的互动为:「结算一次,然后对左侧随从再结算一次并将其冰冻,然后对右侧随从再结算一次并将其冰冻」。
\notice 龙鹰之灵的额外结算也能享受\card{大胆的吞火者}状态的临时英雄技能增强。
\notice 龙鹰之灵的效果是一个光环。也就是说,如果\card{灵体转化}将龙鹰之灵变形,它仍然会再执行两次英雄技能效果。
\notice 在英雄技能额外结算过程中,由于没有触发时机,\card{诺格弗格市长}不会介入。

\subsection{英雄技能使用次数}

判断英雄技能能否使用,是通过比较「本回合已使用英雄技能次数」与「本回合可使用英雄技能次数」来实现的。如果小于则可使用。

你每使用一次英雄技能,「你本回合已使用英雄技能次数」加1。「本回合已使用英雄技能次数」会在英雄技能被替换时时,及回合间清除步骤中清零。

「本回合可使用英雄技能次数」由\term{基本可使用次数}和\term{额外可使用次数}两部分组成。
\begin{itemize}
    \item 一个技能的\term{基本可使用次数}通常为1,\card{要塞指挥官}和\card{考达拉幼龙}分别将技能的基本可使用次数改为2和无穷。
    \item 一个技能的\term{额外可使用次数}通常为0,\card{大富翁比尔杜}、\card{虚空形态}的刷新效果等扳机触发时会使技能的额外可使用次数加1。当你的技能被替换时,额外可使用次数清零。
\end{itemize}
\example 你操控\card{考达拉幼龙},使用\card{援军}两次。然后你使用\card{芬利·莫格顿爵士}将你的英雄技能替换为\card{图腾召唤}。你还可以使用图腾召唤两次。
\notice 即使替换前后的英雄技能是相同的(如通过\card{杂耍吞法者}),相关标签也会清零。
\notice \card{裁决者图哈特}不替换升级的英雄技能(并非替换成相同的升级技能)。因此,它不能重置你的已升级的英雄技能。

此外,英雄技能可以使用还要满足许多其他条件。如场上没有\card{摧心者}、英雄技能不是被动英雄技能等。