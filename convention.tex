\chapter{约定}

\begin{itemize}
    \item 在不产生歧义的情况下,「你受到伤害」「你恢复生命值」「你的生命值/攻击力」等描述中的你均指你的英雄。
    \item 使用\version{X}{}\version{}{Y}\version{X}{Y}表示某条目只在特定版本的游戏中生效。由于测试的不完善,X和Y通常不是准确值。如果X或Y是某个扩展包的缩写,则代表它是这个扩展包的某个未知版本。如果X或Y是「早期版本」,则代表它是某个较早但未知的版本。\version{X}{Y}不包括Y这个版本。
    \item 使用\texttt{e.t}表示实体\texttt{e}上的标签\texttt{t}。使用\texttt{e.t := v}表示「将实体\texttt{e}的标签\texttt{t}设置为\texttt{v}。如果\texttt{e}被省略,则表示实体是唯一的或者显然的。此外,还有一些特殊词语表示特殊的实体:
        \begin{center}
            \begin{tabular}{|c|c|}
                \hline
                \texttt{Game} & 游戏实体 \\
                \hline
                \texttt{Player} & 玩家实体 \\
                \hline
                \texttt{Hero} & 英雄实体 \\
                \hline
            \end{tabular}
        \end{center}
\end{itemize}

在例子中,若无说明,遵循如下约定:

\begin{itemize}
    \item 所有随从和武器都是初始身材和费用。
    \item 双方英雄的生命值足够高。
    \item 没有额外的随从、武器、任务和奥秘。
    \item 所有人的手牌与牌库中的卡牌都对例子无影响。
    \item 所有人的手牌与牌库的数量无关紧要。手牌的上限为十张。
    \item 如果没有描述你操控的随从具体信息,则这些随从的效果与攻击力无关紧要,并且生命值足够高。
\end{itemize}

对于描述「某玩家操控A、B、C」而言,若无说明,遵循如下约定:

\begin{itemize}
    \item A、B、C 的入场顺序与描述顺序相同。
    \item A、B、C 的场上位置无关紧要。
    \item 不同玩家之间实体的入场顺序无关紧要。
\end{itemize}
